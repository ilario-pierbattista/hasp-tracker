\documentclass[a4paper,11pt,oneside]{article}


\usepackage{packages}
%\usepackage{frontespizio}
%\makeindex                                          %CREA INDICE ANALITICO


%Document begin
\begin{document}
	%\pagenumbering{roman} %IMPOSTA NUMERAZIONE ROMANA FINO ALL'INTRODUZIONE
	%\afterpage{\cfoot{\thepage}} %METTE I NUMERI DI PAGINA IN FONDO AL FOOTER. NECESSARIO SE SI USA FANCYHDR
	% \frontespizio
	%\clearpage{\pagestyle{empty}\cleardoublepage}
	%\setlength{\parskip}{0em}
	%\tableofcontents
	%\setlength{\parskip}{1em}
	%\newpage
	%\listoffigures		
	%\newpage
	%\pagenumbering{arabic} %IMPOSTA NUMERAZIONE TRADIZIONALE A PARTIRE DALL'INTRODUZIONE
	%\pagebreak
	%===========================

	\title{Head and Shoulder Profile Detector}
	\author{Ilario Pierbattista}
	\maketitle

	\section{Panoramica generale} % (fold)
	\label{sec:panoramica_generale}
		Si vuole sviluppare un software in grado di riconoscere in modo efficiente e quanto più accurato il profilo umano ripreso dall'alto dal sensore di profondità di un dispositivo Kinect v2.

		\subsection{Hardware e setup} % (fold)
		\label{sub:hardware_e_setup}
			Il sensore di profondità del Kinect ha una risoluzione di 16bit ed una precisione intorno ai 3mm (che diminuisce con l'aumentare della distanza). È montato al soffitto di una stanza a circa tre metri da terra, ha un campo visivo\footnote{FOV: \url{https://en.wikipedia.org/wiki/Field_of_view}} di $70^{\circ} \times 60^{\circ}$, i quale, in relazione alla distanza dal pavimento, permette di catturare un'area di circa $4m \times 5m$.

			Il Kinect cattura 30 frame ogni secondo. La dimensione di ogni frame è pari a $512 \times 424$ pixel.
		% subsection hardware_e_setup (end)

		\subsection{Struttura software} % (fold)
		\label{sub:struttura_software}
			Il software è diviso in due moduli: il modulo per il riconoscimento e il modulo di allenamento.

			Nella prima fase, quella di allenamento, il relativo modulo software, a partire da un insieme di campioni classificati, genererà un insieme di regole per classificare campioni con le medesime caratteristiche. Implementa l'algoritmo Adaboost.

			La seconda fase, quella di riconoscimento, lavora su dati reali: ad ogni frame acquisito tramite il sensore di profondità del Kinect si applicano le regole di classificazione generate nella fase precedente, stabilendo se nel frame è stata ripresa una persona e - in caso affermativo - la posizione di quest'ultima.
		% subsection struttura_software (end)
		
	% section panoramica_generale (end)

	% section prova (end)
\end{document}
