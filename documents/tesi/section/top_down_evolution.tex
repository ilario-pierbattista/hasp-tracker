% !TEX root=../index.tex
\chapter{L'Algoritmo di Allenamento: Adaboost}
\label{chap:adaboost}
    \section{Apprendimento Supervisionato \emph{Ensamble}}
    \label{sec:supervised_ensamble_learning}
        \subsection{Apprendimento Supervisionato}
        \label{sub:supervised_learning}
            \subsubsection{Definizione}
                \paragraph{Obiettivo}
                \paragraph{Spazio delle Ipotesi}
            \subsubsection{Esempi di Supervised learning}
                \paragraph{Algoritmi di Sup.Learning}
                \paragraph{Maggiori campi applicativi}
                \paragraph{Concetti di base}
                    \subparagraph{Overfitting}
            \subsubsection{Ensamble Learning}
        \subsection{\emph{Adaptive Boosting}}
        \label{sub:adaptive_boosting}
            \subsubsection{Algoritmi di Boosting}
            \subsubsection{Aptive: adattabilità}
            \subsubsection{Strong learner e Weak learner}
    \section{Dataset di Allenamento}
    \label{sec:training_dataset}
        \subsection{Categorie di Classificatori}
        \label{sub:classifiers_categories}
            \subsubsection{Variabilità della forma HASP}
                \paragraph{Variazione dell'orientazione}
                \paragraph{Variazione derivata dalla distorsione prospettica}
            \subsubsection{Definizione delle categorie di classificatori}
                \paragraph{Categorie: Verticale e Orizzontale}
                \paragraph{Categorie alternative: Obliquo, a zone}
        \subsection{Preparazione dei Dataset}
        \label{sub:datasets_setup}
            \subsubsection{Acquisizioni}
                \paragraph{Soggetti, percorsi}
                \paragraph{Acquisizione delle registrazioni}
            \subsubsection{Ritaglio dei samples}
                \paragraph{Trainset Creator}
        \subsection{Preprocessing}
        \label{sub:preprocessing}
            \subsubsection{Resize}
                \paragraph{Nearest Neighbour}
                \paragraph{Altri algoritmi di resize}
            \subsubsection{Conversione delle distanze}
    \section{\emph{Strong Learner}}
    \label{sec:strong_learner}
        \subsection{Procedura di estrazione del classificatore forte}
    \section{\emph{Weak Learner}}
    \label{sec:weak_learner}
        \subsection{Procedura di estrazione del classificatore debole}
        \subsection{Valutazione della complessità computazionale}

% Valutare l'accorpamento dei due capitoli successivi
\chapter{Validazione e Regolazione dei Classificatori}
\label{chap:tuning}
    \section{Criteri di Valutazione}
    \label{sec:evaluation_criteria}
    \section{Dataset di Validazione}
    \label{sec:validation_dataset}
        \subsection{Criteri di creazione delle registrazioni}
        \subsection{Altre caratteristiche}
    \section{Massimizzazione all'\emph{Accuracy}}
    \label{sec:accuracy_maximization}
        \subsection{Parametri liberi del classificatore}
            \subsubsection{Numero di weak learner}
            \subsubsection{Soglia del classificatore}
        \subsection{Algoritmo di ricerca della soglia e del NWL ottimi}
    \section{Analisi dei Risultati}
    \label{sec:Analisi dei Risultati}

\chapter{Rilevamento}
\label{chap:rilevamento}
    \section{Tecnica di Rilevamento}
    \label{sec:detection_tecnique}
        \subsection{Detection Window}
        \subsection{Rilevazione su frame}
            \subsubsection{Resize detection window}
            \subsubsection{Slide detection window}
    \section{Selezione della Finestra Migliore} %@TODO
    \label{sec:best_detection_window}
        \subsection{Introduzione al problema}
        \subsection{Algoritmo di selezione}
    \section{Confronto con l'Algoritmo G-C} %@TODO
    \label{sec:gc_algorithm_comparison}

\chapter{Conclusioni}

% Questa parte non è stata ancora analizzata
\begin{appendices}
    \chapter{Software Sviluppato}
    \label{chap:software}
        \section{Componenti}
            \subsection{Creatore dei Dataset}
            \subsection{Allenamento}
            \subsection{Tuning, Testing, Rilevamento}
        \section{Tecnologie utilizzate}
            \subsection{C++ e Matlab}
            \subsection{Git e Github [Opzionale]}
        \section{Proposte di miglioramento}

    \chapter{Cenni del funzionamento del sensore Kinect}
    \label{chap:kinect_sensor}
\end{appendices}
