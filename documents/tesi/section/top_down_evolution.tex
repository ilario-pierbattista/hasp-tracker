% !TEX root=../index.tex

% Capitoli sviluppati fino al 5° livello di profondità
% Introduzione

\chapter{Introduzione}
\label{chap:introduction}
    \section{Human Sensing}
    \label{sec:human_sensing}
        \subsection{Human Sensing}
            \subsubsection{Definizione}
                \paragraph{Definizione enciclopedica di Human Sensing}
                \paragraph{Dispositivi di acquisizione}
                \paragraph{Modalità di riconoscimento (?)}
            \subsubsection{Contesti Applicativi}
                \paragraph{People Detection: Sistemi di sorveglianza}
                \paragraph{People Counting: Indagini di mercato}
                \paragraph{Rilevamento dei parametri vitali: Search & Rescue}
                \paragraph{People Tracking: Ambienti assistivi}
            \subsubsection{HS e Computer Vision}
                \paragraph{Definizione enciclopedica di computer vision}
                \paragraph{Sensori di acquisizione \emph{visivi}}
        \subsection{Stato dell'arte}
            \subsubsection{Pedestrian Detection and Counting}
            \subsubsection{Face Recognition}
            \subsubsection{Kinect: a serious game}
    \section{Panoramica Generale}
    \label{sec:overview}
        \subsection{Introduzione al lavoro di Zhu & Wong}
            \subsubsection{Elenco delle tecnologie coinvolte}
                \paragraph{Kinect per l'acquisizione}
                \paragraph{Adaboost per l'allenamento}
                \paragraph{Affermare la notevole affinità con Viola & Jones}
        \subsection{Configurazione dell'Hardware}
        \label{sub:hardware_configuration}
            \subsubsection{Sensore utilizzato}
                \paragraph{Descrizione sommaria del Kinect}
                \paragraph{Breve esposizione delle caratteristiche tecniche del sensore}
                \paragraph{Presentazione delle immagini di profondità}
            \subsubsection{Configurazione Top-Down}
                \paragraph{Esposizione della configurazione hardware}
                \paragraph{Vantaggi del top-down rispetto al rilevamento frontale}
        \subsection{\emph{Head and Shoulders Profile}}
        \label{sub:hasp}
            \subsubsection{L'attività di riconoscimento è un'attività di classificazione}
                \paragraph{Cosa bisogna riconoscere?}
                \paragraph{Il riconoscimento è un'attività affine alla discriminazione}
                \paragraph{Definizione di classi di oggetti: classificazione}
            \subsubsection{La classificazione si basa sulla misurazione di alcune caratteristiche}
                \paragraph{Discriminare in base a cosa?}
                \paragraph{Le caratteristiche di un oggetto sono la base su cui differenziare un oggetto da un altro}
            \subsubsection{Caratteristiche del profilo HASP in linguaggio naturale}
                \paragraph{Come si può descrivere un'immagine di profondità?}
                \paragraph{Le tre caratteristiche}
        \subsection{Flusso di Lavoro}
        \label{sub:overall_workflow}
            \subsubsection{Definizione dei moduli funzionali}
                \paragraph{Allenamento}
                \paragraph{Rilevamento}
            \subsubsection{Allenamento}
                \paragraph{Input: Dataset}
                \paragraph{Adaboost}
                \paragraph{Output: Classificatore}
            \subsubsection{Rilevamento}
                \paragraph{Input: Frame}
                \paragraph{Algoritmo di rilevamento}
                \paragraph{Output: finestre di rilevamento}

\chapter{Haar-Like Features}
\label{chap:features}
    \section{Definizione}
    \label{sec:haar_features_definition}
        \subsection{Richiamo: cosa è una feature (caratteristica)}
            \subsubsection{Cosa sono le caratteristiche di un oggetto}
            \subsubsection{Le caratteristiche dipendono da cosa si vuole evidenziare}
            \subsubsection{Le caratteristiche dipendono da cosa si ha a disposizione}
        \subsection{Wavelet di Haar}
            \subsubsection{Le feature di Haar derivano dalle wavelet di Haar}
            \subsubsection{Definizione informale delle wavelet di Haar}
                \paragraph{Chi le ha sviluppate}
                \paragraph{Cosa sono (base ortonormale spazio funzionale)}
                \paragraph{Rappresentazione dei segnali (fourier duale)}
            \subsubsection{Wavelet di Haar e DWT}
                \paragraph{Utilizzi commerciali DWT (JPEG2000)}
                \paragraph{Utilizzo delle dwt per il pattern recognition}
        \subsection{Formula di calcolo standard}
            \subsubsection{Rappresentazione visuale}
            \subsubsection{Formula generale}
            \subsubsection{Altri tipi di feature (OpenCv)}
            \subsubsection{Tipi di feature utilizzate}
        \subsection{Cosa mette in evidenza la feature di Haar}
            \subsubsection{Immagini normali (Viola Jones)}
            \subsubsection{Immagini di profondità (Zhu Wong)}
        \subsection{Formula di calcolo invariante ai resize}
            \subsubsection{Anticipazione del problema del ridimensionamento}
            \subsubsection{Formula: Normalizzazione sull'area}
        \subsection{Vantaggi}
            \subsubsection{Differenze di intensità vs Valutazione dei singoli pixel}
            \subsubsection{Differenze di intensità vs Estrazione dei contorni}
            \subsubsection{Estrema efficienza computazionale}
    \section{Immagine Integrale}
    \label{sec:integral_image}
        \subsection{Definizione rigorosa dell'immagine integrale}
            \subsubsection{Problema: efficienza nel calcolo di somme di pixel}
                \paragraph{Complessità computazionale del calcolo \emph{ignorante}}
                \paragraph{Soluzione: rendere queste somme subito disponibili}
            \subsubsection{Definizione immagine integrale}
            \subsubsection{Formula di calcolo della somma dei pixel in un'area}
                \paragraph{Enunciazione della formula}
                \paragraph{Dimostrazione della formula}
                \paragraph{Complessità computazionale del calcolo della feature}
        \subsection{Complessità computazionale generale}
            \subsubsection{Complessità del calcolo dell'immagine integrale}
            \subsubsection{Convenienza del calcolo dell'immagine integrale}
    \section{Decision Stump}
    \label{sec:decision_stump}
        \subsection{Problema: utilizzare le feature}
            \subsubsection{È necessario un meccanismo primitivo per utilizzare le feature}
            \subsubsection{Bisogna discriminare in base al valore}
        \subsection{Definizione di albero decisionale}
        \subsection{Definizione di decision stump}
            \subsubsection{Radice: Test, funzione booleana}
            \subsubsection{Foglie: risultati possibili}
            \subsubsection{Formule di calcolo binaria}
            \subsubsection{Formula di calcolo unica: polarità}

\chapter{L'Algoritmo di Allenamento: Adaboost}
\label{chap:adaboost}
    \section{Apprendimento Supervisionato \emph{Ensamble}}
    \label{sec:supervised_ensamble_learning}
        \subsection{Apprendimento Supervisionato}
        \label{sub:supervised_learning}
            \subsubsection{Definizione}
                \paragraph{Obiettivo}
                \paragraph{Spazio delle Ipotesi}
            \subsubsection{Esempi di Supervised learning}
                \paragraph{Algoritmi di Sup.Learning}
                \paragraph{Maggiori campi applicativi}
                \paragraph{Concetti di base}
                    \subparagraph{Overfitting}
            \subsubsection{Ensamble Learning}
        \subsection{\emph{Adaptive Boosting}}
        \label{sub:adaptive_boosting}
            \subsubsection{Algoritmi di Boosting}
            \subsubsection{Aptive: adattabilità}
            \subsubsection{Strong learner e Weak learner}
    \section{Dataset di Allenamento}
    \label{sec:training_dataset}
        \subsection{Categorie di Classificatori}
        \label{sub:classifiers_categories}
            \subsubsection{Variabilità della forma HASP}
                \paragraph{Variazione dell'orientazione}
                \paragraph{Variazione derivata dalla distorsione prospettica}
            \subsubsection{Definizione delle categorie di classificatori}
                \paragraph{Categorie: Verticale e Orizzontale}
                \paragraph{Categorie alternative: Obliquo, a zone}
        \subsection{Preparazione dei Dataset}
        \label{sub:datasets_setup}
            \subsubsection{Acquisizioni}
                \paragraph{Soggetti, percorsi}
                \paragraph{Acquisizione delle registrazioni}
            \subsubsection{Ritaglio dei samples}
                \paragraph{Trainset Creator}
        \subsection{Preprocessing}
        \label{sub:preprocessing}
            \subsubsection{Resize}
                \paragraph{Nearest Neighbour}
                \paragraph{Altri algoritmi di resize}
            \subsubsection{Conversione delle distanze}
    \section{\emph{Strong Learner}}
    \label{sec:strong_learner}
        \subsection{Procedura di estrazione del classificatore forte}
    \section{\emph{Weak Learner}}
    \label{sec:weak_learner}
        \subsection{Procedura di estrazione del classificatore debole}
        \subsection{Valutazione della complessità computazionale}

% Valutare l'accorpamento dei due capitoli successivi
\chapter{Validazione e Regolazione dei Classificatori}
\label{chap:tuning}
    \section{Criteri di Valutazione}
    \label{sec:evaluation_criteria}
    \section{Dataset di Validazione}
    \label{sec:validation_dataset}
        \subsection{Criteri di creazione delle registrazioni}
        \subsection{Altre caratteristiche}
    \section{Massimizzazione all'\emph{Accuracy}}
    \label{sec:accuracy_maximization}
        \subsection{Parametri liberi del classificatore}
            \subsubsection{Numero di weak learner}
            \subsubsection{Soglia del classificatore}
        \subsection{Algoritmo di ricerca della soglia e del NWL ottimi}
    \section{Analisi dei Risultati}
    \label{sec:Analisi dei Risultati}

\chapter{Rilevamento}
\label{chap:rilevamento}
    \section{Tecnica di Rilevamento}
    \label{sec:detection_tecnique}
        \subsection{Detection Window}
        \subsection{Rilevazione su frame}
            \subsubsection{Resize detection window}
            \subsubsection{Slide detection window}
    \section{Selezione della Finestra Migliore} %@TODO
    \label{sec:best_detection_window}
        \subsection{Introduzione al problema}
        \subsection{Algoritmo di selezione}
    \section{Confronto con l'Algoritmo G-C} %@TODO
    \label{sec:gc_algorithm_comparison}

\chapter{Conclusioni}

% Questa parte non è stata ancora analizzata
\begin{appendices}
    \chapter{Software Sviluppato}
    \label{chap:software}
        \section{Componenti}
            \subsection{Creatore dei Dataset}
            \subsection{Allenamento}
            \subsection{Tuning, Testing, Rilevamento}
        \section{Tecnologie utilizzate}
            \subsection{C++ e Matlab}
            \subsection{Git e Github [Opzionale]}
        \section{Proposte di miglioramento}

    \chapter{Cenni del funzionamento del sensore Kinect}
    \label{chap:Cenni del funzionamento del sensore Kinect}
\end{appendices}
