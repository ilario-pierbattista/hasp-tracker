% !TEX root=../index.tex

% Capitoli sviluppati fino al 5° livello di profondità
% Introduzione

\chapter{Introduzione}
\label{chap:introduction}
    \section{Human Sensing}
    \label{sec:human_sensing}
        \subsection{Human Sensing}
            \subsubsection{Definizione}
                \paragraph{Definizione enciclopedica di Human Sensing}
                \paragraph{Dispositivi di acquisizione}
                \paragraph{Modalità di riconoscimento (?)}
            \subsubsection{Contesti Applicativi}
                \paragraph{People Detection: Sistemi di sorveglianza}
                \paragraph{People Counting: Indagini di mercato}
                \paragraph{Rilevamento dei parametri vitali: Search & Rescue}
                \paragraph{People Tracking: Ambienti assistivi}
            \subsubsection{HS e Computer Vision}
                \paragraph{Definizione enciclopedica di computer vision}
                \paragraph{Sensori di acquisizione \emph{visivi}}
        \subsection{Stato dell'arte}
            \subsubsection{Pedestrian Detection and Counting}
            \subsubsection{Face Recognition}
            \subsubsection{Kinect: a serious game}
    \section{Panoramica Generale}
    \label{sec:overview}
        \subsection{Introduzione al lavoro di Zhu & Wong}
            \subsubsection{Elenco delle tecnologie coinvolte}
                \paragraph{Kinect per l'acquisizione}
                \paragraph{Adaboost per l'allenamento}
                \paragraph{Affermare la notevole affinità con Viola & Jones}
        \subsection{Configurazione dell'Hardware}
        \label{sub:hardware_configuration}
            \subsubsection{Sensore utilizzato}
                \paragraph{Descrizione sommaria del Kinect}
                \paragraph{Breve esposizione delle caratteristiche tecniche del sensore}
                \paragraph{Presentazione delle immagini di profondità}
            \subsubsection{Configurazione Top-Down}
                \paragraph{Esposizione della configurazione hardware}
                \paragraph{Vantaggi del top-down rispetto al rilevamento frontale}
        \subsection{\emph{Head and Shoulders Profile}}
        \label{sub:hasp}
            \subsubsection{L'attività di riconoscimento è un'attività di classificazione}
                \paragraph{Cosa bisogna riconoscere?}
                \paragraph{Il riconoscimento è un'attività affine alla discriminazione}
                \paragraph{Definizione di classi di oggetti: classificazione}
            \subsubsection{La classificazione si basa sulla misurazione di alcune caratteristiche}
                \paragraph{Discriminare in base a cosa?}
                \paragraph{Le caratteristiche di un oggetto sono la base su cui differenziare un oggetto da un altro}
            \subsubsection{Caratteristiche del profilo HASP in linguaggio naturale}
                \paragraph{Come si può descrivere un'immagine di profondità?}
                \paragraph{Le tre caratteristiche}
        \subsection{Flusso di Lavoro}
        \label{sub:overall_workflow}
            \subsubsection{Definizione dei moduli funzionali}
                \paragraph{Allenamento}
                \paragraph{Rilevamento}
            \subsubsection{Allenamento}
                \paragraph{Input: Dataset}
                \paragraph{Adaboost}
                \paragraph{Output: Classificatore}
            \subsubsection{Rilevamento}
                \paragraph{Input: Frame}
                \paragraph{Algoritmo di rilevamento}
                \paragraph{Output: finestre di rilevamento}

\chapter{Haar-Like Features}
\label{chap:features}
    \section{Definizione}
    \label{sec:haar_features_definition}
        \subsection{Richiamo: cosa è una feature (caratteristica)}
            \subsubsection{Cosa sono le caratteristiche di un oggetto}
            \subsubsection{Le caratteristiche dipendono da cosa si vuole evidenziare}
            \subsubsection{Le caratteristiche dipendono da cosa si ha a disposizione}
        \subsection{Wavelet di Haar}
            \subsubsection{}
        \subsection{Formula di calcolo standard}
        \subsection{Cosa mette in evidenza la feature di Haar}
        \subsection{Formula di calcolo invariante ai resize}
    \section{Immagine Integrale}
    \label{sec:integral_image}
    \section{Decision Stump}
    \label{sec:decision_stump}

\chapter{L'Algoritmo di Allenamento: Adaboost}
\label{chap:adaboost}
    \section{Apprendimento Supervisionato \emph{Ensamble}}
    \label{sec:supervised_ensamble_learning}
        \subsection{Apprendimento Supervisionato}
        \label{sub:supervised_learning}
        \subsection{\emph{Ensamble Learning}}
        \label{sub:ensamble_learning}
        \subsection{\emph{Adaptive Boosting}}
        \label{sub:adaptive_boosting}
    \section{Dataset di Allenamento}
    \label{sec:training_dataset}
        \subsection{Categorie di Classificatori}
        \label{sub:classifiers_categories}
        \subsection{Preparazione dei Dataset}
        \label{sub:datasets_setup}
        \subsection{Preprocessing}
        \subsection{Formula di calcolo standard}
        \subsection{Cosa mette in evidenza la feature di Haar}
        \subsection{Formula di calcolo invariante ai resize}
    \section{Immagine Integrale}
    \label{sec:integral_image}
    \section{Decision Stump}
    \label{sec:decision_stump}

\chapter{L'Algoritmo di Allenamento: Adaboost}
\label{chap:adaboost}
    \section{Apprendimento Supervisionato \emph{Ensamble}}
    \label{sec:supervised_ensamble_learning}
        \subsection{Apprendimento Supervisionato}
        \label{sub:supervised_learning}
        \subsection{\emph{Ensamble Learning}}
        \label{sub:ensamble_learning}
        \subsection{\emph{Adaptive Boosting}}
        \label{sub:adaptive_boosting}
    \section{Dataset di Allenamento}
    \label{sec:training_dataset}
        \subsection{Categorie di Classificatori}
        \label{sub:classifiers_categories}
        \subsection{Preparazione dei Dataset}
        \label{sub:datasets_setup}
        \subsection{Preprocessing}
        \label{sub:preprocessing}
    \section{\emph{Strong Learner}}
    \label{sec:strong_learner}
    \section{\emph{Weak Learner}}
    \label{sec:weak_learner}

\chapter{Validazione e Regolazione dei Classificatori}
\label{chap:tuning}
    \section{Criteri di Valutazione}
    \label{sec:evaluation_criteria}
    \section{Dataset di Validazione}
    \label{sec:validation_dataset}
    \section{Massimizzazione all'\emph{Accuracy}}
    \label{sec:accuracy_maximization}
    \section{Analisi dei Risultati}
    \label{sec:Analisi dei Risultati}

\chapter{Rilevamento}
\label{chap:rilevamento}
    \section{Tecnica di Rilevamento}
    \label{sec:detection_tecnique}
    \section{Selezione della Finestra Migliore} %@TODO
    \label{sec:best_detection_window}
    \section{Confronto con l'Algoritmo G-C} %@TODO
    \label{sec:gc_algorithm_comparison}

\chapter{Conclusioni}

% Questa parte non è stata ancora analizzata
\begin{appendices}
    \chapter{Struttura Software}
    \label{chap:Struttura Software}

    \chapter{Cenni del funzionamento del sensore Kinect}
    \label{chap:Cenni del funzionamento del sensore Kinect}
\end{appendices}
