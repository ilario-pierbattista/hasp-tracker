% !TEX root=../index.tex

\chapter*{Introduzione}
\addcontentsline{toc}{chapter}{Introduzione}
\label{cap:Introduzione}
Il monitoraggio delle persone è un problema di visione artificiale di importanza fondamentale.
Sistemi di \emph{human sensing} vengono continuamente sviluppati ed utilizzati nei più disparati contesti applicativi. Sistemi di sorveglianza, apparecchiatura di supporto per missioni di salvataggio, dispositivi designati all'utilizzo in ambienti assistivi automatizzati e persino alcuni sistemi automatici per effettuare indagini di mercato utilizzano tecniche di percezione automatica delle persone dall'elaborazione dei dati acquisiti per mezzo di sensori.

Nel 2013 Zhu e Wong descrivono in \cite{Zhu13} un sistema allenabile di rilevamento e conteggio delle persone che attraversano una stanza.
Il riconoscimento della persona avviene elaborando i dati catturati dal sensore di profondità di un dispositivo Kinect, il quale è montato sul soffitto della stanza ed è puntato verso il pavimento. La crescente popolarità dei dispositivi Kinect, anche al di fuori degli ambienti videoludici, lo rende un interessante oggetto di studio.

Il presente elaborato fornisce una dettagliata analisi riguardante l'implementazione e le caratteristiche di tale sistema di rilevamento.
Nel capitolo 1 si parlerà di tizio, Nel capitolo due si parlerà di caio.
