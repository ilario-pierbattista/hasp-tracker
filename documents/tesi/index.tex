\documentclass[a4paper,11pt,oneside]{book}
\synctex=1


\usepackage{packages}
\usepackage{frontespizio}
%\makeindex                                          %CREA INDICE ANALITICO


%Document begin
\begin{document}
	%\pagenumbering{roman} %IMPOSTA NUMERAZIONE ROMANA FINO ALL'INTRODUZIONE
	%\afterpage{\cfoot{\thepage}} %METTE I NUMERI DI PAGINA IN FONDO AL FOOTER. NECESSARIO SE SI USA FANCYHDR
	\frontespizio
	%\clearpage{\pagestyle{empty}\cleardoublepage}
	%\setlength{\parskip}{0em}
	\tableofcontents
	%\setlength{\parskip}{1em}
	%\newpage
	%\listoffigures
	\newpage
	%\pagenumbering{arabic} %IMPOSTA NUMERAZIONE TRADIZIONALE A PARTIRE DALL'INTRODUZIONE
	%\pagebreak
	%===========================

	% Numeri vicino alla riga
	% @TODO Rimuovere dopo le correzioni
	\linenumbers

	\onehalfspacing

	% % !TEX root=../index.tex
\section{Mindmap}
\label{sec:Mindmap}

\begin{tikzpicture}
    \path[mindmap,concept color=black,text=black]

    node[concept] {Hasp detection.}
    [clockwise from=0]

    child[concept] {
        node[concept] {Classificazione}
    }

    child[concept] {
        node[concept] {practical}
        [clockwise from=90]
        child { node[concept] {algorithms} }
        child { node[concept] {data structures} }
        child { node[concept] {programming languages} }
        child { node[concept] {software engineering} }
    }
    child[concept] {
        node[concept] {applied}
        [clockwise from=-30]
        child { node[concept] {databases} }
        child { node[concept] {WWW} }
    }
    ;
\end{tikzpicture}

	% \newpage

	% Introduzione
	% !TEX root=../index.tex

\chapter{Introduzione}
\label{cap:Introduzione}
\emph{Presentazione del problema, Stato dell'arte e descrizione delle caratteristiche peculiari del problema di riconoscimento.
Cenni sul funzionamento del sensore Kinect.
Head and Shoulder Profile.
Analisi del problema di classificazione.}

Il monitoraggio delle persone è un problema di visione artificiale di importanza fondamentale.
Sistemi di \emph{human sensing} vengono continuamente sviluppati ed utilizzati nei più disparati contesti applicativi. Sistemi di sorveglianza, apparecchiatura di supporto per missioni di salvataggio, dispositivi designati all'utilizzo in ambienti assistivi automatizzati e persino alcuni sistemi automatici per effettuare indagini di mercato utilizzano tecniche di percezione automatica delle persone dall'elaborazione dei dati acquisiti per mezzo di sensori.

Nel 2013 Zhu e Wong descrivono in \cite{Zhu13} un sistema allenabile di rilevamento e conteggio delle persone che attraversano una stanza.
Il riconoscimento della persona avviene elaborando i dati catturati dal sensore di profondità di un dispositivo Kinect, il quale è montato sul soffitto della stanza ed è puntato verso il pavimento. La crescente popolarità dei dispositivi Kinect, anche al di fuori degli ambienti videoludici, lo rende un interessante oggetto di studio.

\section{Head and Shoulder Profile}
\label{sec:hasp}
\begin{wrapfigure}{L}{0cm}
    \centering
    \includegraphics[width=5cm]{img/no_occlusion.png}
    \caption{Due persone in un ritaglio proveniente da un frame di profondità. Il frame è stato acquisito con il Kinect V1, a differenza di quello in figura \ref{fig:spatial_feature}, acquisito con il Kinect V2.}
    \label{fig:no_occlusion}
\end{wrapfigure}

In condizioni ottimali, la figura umana ripresa dall'alto è composta solamente dalla testa e dalle spalle.
Finchè si trova quasi in corrispondenza del sensore, nella parte centrale dell'immagine di profondità, la restante parte del corpo rimane quasi del tutto nascosta. Sulla posizione delle braccia non si possono fare delle assunzioni precise.

In figura \ref{fig:no_occlusion}, la figura dell'individuo a destra corrisponde a tale descrizione, mentre dell'altro è visibile parte del corpo ed una delle due spalle è nascosta dalla testa.
Nelle zone periferiche di un frame di profondità, l'immagine è soggetta alla \emph{distorsione prospettica} così come lo è quella di qualunque telecamere RGB.

Tale distorsione costituisce un disturbo, dal momento che la figura dello stesso soggetto varia a seconda della relativa posizione nell'area di visualizzazione.
Si vedrà più avanti come affrontare tale situazione, per il momento si considerano solamente le immagini proveniente dalle zone centrali dei frame (come quella in figura \ref{fig:spatial_feature}).

Un grande vantaggio dell'utilizzo del Kinect in questa configurazione è l'\emph{assenza di occlusione}.
Infatti, rispetto ai molteplici sistemi di riconoscimento frontali, i soggetti non possono nascondersi l'uno con l'altro al sensore.

Utilizzare le immagini di profondità significa ragionare con le distanze: piuttosto che cercare di descrivere l'immagine del profilo umano in termini di forma, deve essere descritto in termini di differenze di quota rispetto all'ambiente circostante.
Alla luce di ciò, si possono identificare i seguenti criteri descrittivi:

\begin{enumerate}
    \item L'immagine di una persona è caratterizzata da uno spazio vuoto di fronte ad essa e dietro di essa. Per \emph{spazio vuoto} si intende una regione la cui distanza dal sensore è circa quella del pavimento.
    \item A sinistra della spalla sinistra ed a destra della spalla destra del profilo dall'alto di una persona, sono presenti degli spazi vuoti.
    \item Tra la testa e le spalle vi è una differenza di altezza.

\end{enumerate}

\begin{wrapfigure}{R}{0cm}
    \centering
    \includegraphics[width=4cm]{img/spatial_features.png}
    \caption{Una persona mentre cammina}
    \label{fig:spatial_feature}
\end{wrapfigure}

Di qui in avanti, con l'acronimo \emph{HASP} (\emph{Head And Shoulder Profile}), ci si riferirà proprio al profilo della persona ripreso dall'alto che soddisfa i criteri appena elencati.


	% !TEX root=../index.tex

\chapter{Panoramica del sistema}
\label{cap:overview}
Riproducendo la configurazione proposta in \cite{Zhu13} al soffitto del laboratorio è stato fissato un dispositivo Kinect V2 rivolto verso il pavimento: all'ingresso di una persona nella visuale del Kinect, ne saranno ben visibili la testa e le spalle.

\section{Il Kinect}
\label{sec:sensor}
Ad onor del vero, ciò che viene comunemente chiamato \emph{sensore di profondità} (o sensore di distanza) del Kinect, è in realtà uno \emph{scanner 3D a luce strutturata}\footnote{Nonostante ciò, lo si continuerà a chiamare \emph{sensore di profondità}, leggermente inesatto, ma decisamente più breve ed intuitivo}.
\begin{wrapfigure}{R}{0cm}
    \centering
    \includegraphics[width=8cm]{img/3d-structured-light-scanner.png}
    \label{fig:structured_light_scanner}
    \caption{Schematizzazione di uno scanner 3D a luce strutturata.}
\end{wrapfigure}
Un sorgente di raggi infrarossi proietta una serie di pattern codificati. La deformazione indotta dalle superfici degli oggetti interessati viene acquisita da una o più telecamere ed utilizzata per il calcolo delle coordinate tridimensionali.

Il risultato di un sensore di questo tipo è un insieme di triplette $(x,y,z)$, organizzate in una \emph{immagine di profontità}, una struttura dati che è molto simile ad una semplice immagine in scala dei grigi, dove il valore di ogni pixel rappresenta la misura in millimetri della distanza della superficie dal sensore.
La forte somiglianza con le immagini in scala dei grigi è supportata dal fatto che ogni pixel è codificato utilizzando 16bit.

La massima affidabilità del sensore del Kinect si ha per distanza comprese tra $50cm$ e $4,5m$.
Il dispositivo è montato al soffitto a $2,8m$ da terra e ha un campo visivo di $70^{\circ} \times 60^{\circ}$, il quale, all'altezza del pavimento, determina un'area di cattura di circa $4m \times 5m$.

La dimensione di ogni immagine di profondità è di $512 \times 424$ pixel. Nativamente non vengono codificate in alcun modo particolare, sono delle semplici matrici di interi.
Il Kinect V2 è in grado di catturarne fino a 30 al secondo. Utilizzando un apposito software di registrazione è stato possibile mettere insieme dei video di profondità a 30 fps.


\section{Head and Shoulder Profile}
\label{sec:hasp}

\begin{wrapfigure}{L}{0cm}
    \centering
    \includegraphics[width=5cm]{img/no_occlusion.png}
    \caption{Due persone in un ritaglio proveniente da un frame di profondità. Il frame è stato acquisito con il Kinect V1, a differenza di quello in figura \ref{fig:spatial_feature}, acquisito con il Kinect V2.}
    \label{fig:no_occlusion}
\end{wrapfigure}

In condizioni ottimali, la figura umana ripresa dall'alto è composta solamente dalla testa e dalle spalle.
Finchè si trova quasi in corrispondenza del sensore, nella parte centrale dell'immagine di profondità, la restante parte del corpo rimane quasi del tutto nascosta. Sulla posizione delle braccia non si possono fare delle assunzioni precise.

In figura \ref{fig:no_occlusion}, la figura dell'individuo a destra corrisponde a tale descrizione, mentre dell'altro è visibile parte del corpo ed una delle due spalle è nascosta dalla testa.
Nelle zone periferiche di un frame di profondità, l'immagine è soggetta alla \emph{distorsione prospettica} così come lo è quella di qualunque telecamere RGB.

Tale distorsione costituisce un disturbo, dal momento che la figura dello stesso soggetto varia a seconda della relativa posizione nell'area di visualizzazione.
Si vedrà più avanti come affrontare tale situazione, per il momento si considerano solamente le immagini proveniente dalle zone centrali dei frame (come quella in figura \ref{fig:spatial_feature}).

Un grande vantaggio dell'utilizzo del Kinect in questa configurazione è l'\emph{assenza di occlusione}.
Infatti, rispetto ai molteplici sistemi di riconoscimento frontali, i soggetti non possono nascondersi l'uno con l'altro al sensore.

Utilizzare le immagini di profondità significa ragionare con le distanze: piuttosto che cercare di descrivere l'immagine del profilo umano in termini di forma, deve essere descritto in termini di differenze di quota rispetto all'ambiente circostante.
Alla luce di ciò, si possono identificare i seguenti criteri descrittivi:

\begin{enumerate}
    \item L'immagine di una persona è caratterizzata da uno spazio vuoto di fronte ad essa e dietro di essa. Per \emph{spazio vuoto} si intende una regione la cui distanza dal sensore è circa quella del pavimento.
    \item A sinistra della spalla sinistra ed a destra della spalla destra del profilo dall'alto di una persona, sono presenti degli spazi vuoti.
    \item Tra la testa e le spalle vi è una differenza di altezza.

\end{enumerate}

\begin{wrapfigure}{R}{0cm}
    \centering
    \includegraphics[width=4cm]{img/spatial_features.png}
    \caption{Una persona mentre cammina}
    \label{fig:spatial_feature}
\end{wrapfigure}

Di qui in avanti, con l'acronimo \emph{HASP} (\emph{Head And Shoulder Profile}), ci si riferirà proprio al profilo della persona ripreso dall'alto che soddisfa i criteri appena elencati.


\section{Un Problema di Classificazione}
\label{sec:classification_problem}
Quello che il sistema di rilevamento deve fare è riconoscere se e dove è presente l'immagine HASP di una persona all'interno del frame di profondità.
Ciò avviene analizzando sequenzialmente delle porzioni del frame originale in modo da coprire tutta l'area. Si tornerà più avanti su questo aspetto, per il momento si consideri che il problema di rilevamento è naturalmente riconducibile ad un problema di classificazione.

Gli oggetti da classificare sono immagini di profondità e le possibili classi di appartenenza di tali oggetti sono due: la classe delle immagini che ritraggono il profilo di una persona è la classe delle immagini che invece ne sono prive.
Il fatto che vi siano solamente due classi lo rende un \emph{problema di classificazione binario}.

L'operazione di classificazione avviene mediante la misurazione di alcune \emph{caratteristiche} (\emph{feature}) interessanti dell'immagine di profondità.
I criteri di riconoscimento presentati nella sezione \ref{sec:hasp} sono dei validi esempi di caratteristiche. Sono espresse in linguaggio naturale e sono comprensibili agli esseri umani, ma mancano di rigore e perciò non sono direttamente implementabili come criteri di classificazione delle immagini HASP.

Il \emph{classificatore} è il componente che esegue effettivamente la classificazione. La definizione di un classificatore può avvenire \emph{manualmente}, mediante l'applicazione di un modello atto a descrivere al meglio gli oggetti di una determinata classe, oppure può essere \emph{allenato}.
Gli algoritmi per la definizione di classificatori allenati si distinguono a loro volta tra algoritmi di \emph{allenamento supervisionato} e di \emph{allenamento non supervisionato}, a seconda che facciano uso o meno di un insieme di allenamento.

Un insieme di allenamento non è altro che un insieme di oggetti adeguati al problema di classificazione per cui viene fornita la classificazione \emph{reale}.
In questo caso saranno delle immagini di profondità opportunamente marcate come immagini che contengono o meno il profilo di una persona.

Il sistema in esame prevede l'utilizzo di classificatori allenati con \emph{Adaboost}. Sono molti i sistemi di riconoscimento di immagini che utilizzano Adaboost per allenare i rispettivi classificatori, primo tra tutti il quello di riconoscimento facciale proposto da Viola e Jones \cite{Viola04}, ad oggi considerato il più robusto ed efficiente della sua categoria.

La forte somiglianza con il lavoro di Viola e Jones farà sì che saranno molteplici i riferimenti ad esso nel testo e i dettagli implementativi in comune.


\section{Preprocessing}
\label{sec:preprocessing}
Ogni pixel di un'immagine di profondità denota la distanza in millimetri della superficie dell'oggetto dal sensore. Tuttavia, prima di poter essere utilizzate da qualsiasi componente del sistema di rilevamento, occorre convertire il valore di ogni pixel in quota della superficie dal pavimento.

Il pavimento è la superficie la cui distanza dal sensore è massima, quindi per individuarne il valore sarebbe sufficiente ricercare il valore massimo di ciascuna immagine di profondità.
Nella pratica ciò non è vero, in quanto, a causa della distorsione prospettica, nelle aree periferiche dell'immagine la distanza percepita del pavimento è maggiore rispetto a quella misurata al centro dell'immagine.

Quindi, per determinare la reale distanza del pavimento è necessario disporre di un'immagine di riferimento che ritragga solamente il pavimento stesso. A questo punto la distanza di riferimento è quella misurata esattamente al centro del frame.

La trasformazione delle distanze in quote avviene, come è intuibile, sostituendo il valore di ogni pixel $d$ con il valore $d'$.
\begin{equation}
    d' = d - d_{pavimento}
    \label{eq:floor_distance}
\end{equation}
L'operazione di conversione ha complessità computazionale $\Theta(n \cdot m)$, dove $n$ ed $m$ sono le dimensioni del frame.


	% !TEX root=../index.tex

\section{Weak learner}
\label{sec:weak_learner}
\begin{enumerate}
    \item Problema di classificazione
    \item Caratteristiche
\end{enumerate}

\subsection{Feature Haar-like}
\label{sub:feature_haar_like}

\subsubsection{Immagine Integrale}
\label{subs:immagine_integrale}

\subsection{Decision Stump}
\label{subs:decision_stump}


	\input{section/adaboost.tex}

	\chapter{Altro}
	\section{Costruzione dei Dataset}
	\label{sec:Costruzione dei Dataset}

	\section{Test e valutazione}
	\label{sec:Test e valutazione}

	\section{Rilevazioni su registrazioni reali}
	\label{sec:Rilevazioni su registrazioni reali}

	\section{Conclusioni}
	\label{sec:Conclusioni}

	% Far comparire tutti gli elementi della bibliografia
	\nocite{*}
	\bibliographystyle{plainnat}
	\bibliography{bibliografia}{}

\end{document}
