\documentclass[a4paper,11pt,oneside]{book}
\synctex=1

\usepackage{packages}
\usepackage{frontespizio}
%\makeindex                                          %CREA INDICE ANALITICO

%Document begin
\begin{document}
	%\pagenumbering{roman} %IMPOSTA NUMERAZIONE ROMANA FINO ALL'INTRODUZIONE
	%\afterpage{\cfoot{\thepage}} %METTE I NUMERI DI PAGINA IN FONDO AL FOOTER. NECESSARIO SE SI USA FANCYHDR
	\frontespizio
	%\clearpage{\pagestyle{empty}\cleardoublepage}
	%\setlength{\parskip}{0em}
	\tableofcontents
	%\setlength{\parskip}{1em}
	%\newpage
	%\listoffigures
	\newpage
	%\pagenumbering{arabic} %IMPOSTA NUMERAZIONE TRADIZIONALE A PARTIRE DALL'INTRODUZIONE
	%\pagebreak
	%===========================

	% Numeri vicino alla riga
	% @TODO Rimuovere dopo le correzioni
	\linenumbers

	% \onehalfspacing

	% Introduzione
	% !TEX root=../index.tex

\chapter{Introduzione}
\label{chap:introduction}
    \section{Human Sensing}
    \label{sec:human_sensing}
        L'insieme di tecniche e di soluzioni per il riconoscimento della presenza della persona nello spazio prende il nome \emph{human sensing}.

        Vengono utilizzati vari dispositivi di sensoristica per l'acquisizione di informazioni ai fini del riconoscimento. 
        Dall'elaborazione delle informazioni acquisite, i sistemi di human sensing sono in grado di rilevare la presenza e la posizione della persona all'interno dell'ambiente interessato.

        I contesti applicativi di sistemi di questo genere sono variegati ed in costante espansione.
        Sistemi per il rilevamento delle persone si adattano perfettamente in contesti di sorveglianza, sia al fine di garantire la sicurezza di un ambiente in termini di rilevamento delle intrusioni, sia al fine di costruire delle soluzioni di monitoraggio in ambienti assistivi automatizzati.
        Dispositivi in grado di localizzare corpi umani sono utilissimi nelle attività \emph{search \& rescue}, dove è necessario, in caso di catastrofi o calamità naturali, localizzare nel minor tempo possibile i superstiti in condizioni avverse.
        Vi sono delle applicazioni anche in ambito economico: soluzioni di \emph{people counting} sono sempre più usate nei processi di \emph{retail} per effettuare analisi di mercato.

        Molto spesso, la natura dei sensori utilizzati per le acquisizioni, fanno in modo che i problemi di human sensing si intersechino con problemi di \emph{computer vision}: l'utilizzo di sensori ad acquisizione visiva comporta l'insorgere di svariati problemi di visione artificiale.

        Lo scopo della visione artificiale è quello di riprodurre la vista umana, intesa non come semplice acquisizione di rappresentazione bidimensionale di una regione di spazio, ma mira ad una reale interpretazione del relativo contenuto.

        \subsection{Stato dell'Arte}
            Il crescente interesse per le soluzioni di human sensing ha dato luogo a numerosi studi e ricerche sull'argomento.
            Rimanendo nel contesto del rilevamento legato a problemi di computer vision, spiccano i seguenti risultati.

            \begin{description}
                \item[People Tracking and Counting] \citet{Papageorgiou98} nel \citeyear{Papageorgiou98} proposero una tecnica di riconoscimento e conteggio di pedoni a partire da immagini RGB.
                Elaborando opportunamente le informazioni raccolte da comuni telecamere, il sistema era in grado di riconoscere il pattern della figura umana analizzando le variazioni di intensità tra le diverse aree delle immagini.
                Per l'apprendimento utilizzarono una \emph{macchina a vettori di supporto}\footnote{Consultare la sezione \ref{sec:supervised_ensamble_learning} ed in generale tutto il capitolo \ref{chap:adaboost} per ulteriori informazioni al riguardo.}.
                
                \item[Face Recognition] Viola e Jones in \cite{Viola04} descrivono un framework di \emph{object recognition}, applicandolo con successo al caso del riconoscimento dei volti.
                Ancora una volta, si tratta di un sistema allenabile che utilizza l'algoritmo \emph{Adaboost}\footnote{Per maggiori informazioni consultare \cite{Freund97} ed il capitolo \ref{chap:adaboost}.}.

                \item[Human Sensing tramite Kinect] Anche il dispositivo Kinect, prodotto da Microsoft, costituisce, nel complesso, un valido esempio di sistema di human sensing.
                La quantità e la qualità dei sensori con cui è equipaggiato, il costo relativamente basso e la disponibilità di \emph{SDK} e toolkit integrabili con ambienti di sviluppo evoluti, stanno accrescendo la popolarità di tale dispositivo al di là delle applicazioni videoludiche.
                Cresce anche l'interesse per i \emph{serious game}, videogiochi sviluppati con una doppia valenza, quella di assistere e guidare i pazienti in contesti riabilitativi.
            \end{description}

    \section{Panoramica Generale}
    \label{sec:overview}
        Questo elaborato descrive principalmente il lavoro svolto per l'implementazione del sistema di rilevamento descritto da Zhu e Wong in \cite{Zhu13}.
        È correlato di numerosi approfondimenti teorici, necessari, sia per avere una corretta visione d'insieme del problema, sia per essere in grado di agire efficacemente nella futura richiesta di miglioramenti nell'implementazione proposta.

        \subsection{Il Lavoro di Zhu e Wong} % (fold)
        \label{sub:il_lavoro_di_zhu_e_wong}
            Nel \citeyear{Zhu13} Zhu e Wong presentano un sistema di riconoscimento della figura umana, ripresa dall'alto di una stanza, con il dispositivo Kinect.
            Si tratta di una soluzione di human sensing, intercalata in una particolare condizione ambientale, che utilizza il \emph{sensore di profondità} del dispositivo Kinect come sorgente per l'acquisizione di dati.

            Come altre soluzioni proposte in precedenza, il sistema deve essere allenato a riconoscere persone.
            L'algoritmo utilizzato per lo sviluppo dei criteri di riconoscimento è \emph{Adaboost}, lo stesso utilizzato per il framework di \emph{face detection} sviluppato da Viola e Jones.
            Come si potrà vedere in seguito, in maniera più dettagliata, saranno numerose le somiglianze con quest'ultimo.

        \subsection{Configurazione Hardware}
        \label{sub:hardware_configuration}
            Precisamente la versione del dispositivo Kinect utilizzato è la \emph{V2}.
            Il sensore di profondità del Kinect V2 fornisce una rappresentazione bidimensionale dello spazio, fondamentalmente sotto forma di immagini particolari. 
            In queste ultime ogni pixel corrisponde il valore in millimetri della distanza dal sensore della superficie dell'oggetto presente nell'immagine.
            Ci si riferirà ad esse chiamandole \emph{immagini di profondità}.

            Il sensore del Kinect, di cui è disponibile una piccola descrizione più dettagliata all'appendice \ref{chap:kinect}, fornisce uno stream di tali immagini ad una frequenza di 30 frame al secondo: è possibile quindi registrare dei \emph{video di profondità}.

            \subsubsection{Visuale Top-Down}
                Il dispositivo Kinect viene montato al soffitto di una stanza e l'ambiente viene ripreso da tale prospettiva.
                Solitamente l'altezza a cui viene montato è di poco inferiore alla distanza del soffitto dal pavimento (poco meno di $3m$), inoltre la linea focale del sensore dovrebbe essere quanto più possibile ortogonale al pavimento della stanza, in modo da ridurre ai minimi termini la presenza di asimmetrie nelle riprese.
                
                Tali distanze sono perfettamente compatibili con le specifiche tecniche del dispositivo stesso.
                Nel caso in cui vi sia la necessità di montare il Kinect a soffitti più alti di $4m$, si possono utilizzare delle lenti correttive per aumentare il range di affidabilità del sensore.

                Molti sistemi di riconoscimento utilizzano il Kinect - e più in generale, qualsiasi dispositivo di acquisizione visivo - in posizione frontale rispetto ai soggetti da riconoscere.
                Infatti, il dispositivo, concepito per applicazioni videoludiche, è progettato per operare in tale posizione.
                Tuttavia, la configurazione descritta precedentemente, ha l'enorme vantaggio di eliminare l'occlusione del soggetto: in tali condizioni una persona non può nasconderne dietro di sè un'altra alla vista del sensore (se non in scomode posizioni), cosa invece frequentissima nei i sistemi di rilevamento frontali.

        \subsection{\emph{Head and Shoulders Profile}}
        \label{sub:hasp}
            Riconoscere un oggetto significa mettere in relazione la sua rappresentazione con un concetto, più o meno specifico, che lo descriva.
            In questo caso, dovendo riconoscere delle persone, si dovrà mettere correttamente in relazione un'immagine di profondità che raffigura un individuo con il \emph{concetto di persona}, mentre per un'immagine che non la raffigura, si dovrà evitare una relazione di tale natura.
            
            Si definiscono due classi.
            Il concetto di classe è molto simile a quello delle \emph{classi di equivalenza}, relativo all'algebra astratta, ma anche a quello delle classi come prototipi di \emph{oggetti software}, relativo al paradigma di programmazione ad oggetti.
            Una classe costituisce un insieme di oggetti che condividono determinate proprietà caratteristiche: 
            per le classi di equivalenza, tutti gli elementi appartenenti ad essa devono essere tra di loro in una relazione di equivalenza;
            per le classi software, tutte istanze di tali classi sono caratterizzate dalla medesima struttura (stessi attributi, stessi metodi), anche se poi ogni oggetto è ben distinto dall'altro. 

            Distinguere gli oggetti che rappresentano persone da quelli che non le rappresentano, significa classificare tali oggetti in due classi: la \emph{classe delle persone} e quella delle \emph{non persone}.
            A questo punto, bisogna iniziare a chiedersi quali debbano essere le proprietà degli elementi dell'una e quali quelli dell'altra.

            Bisogna sottolineare il fatto che gli oggetti di una stessa classe, hanno sì delle proprietà in comune tra di loro, ma ne hanno altre per cui differiscono.
            Individuare le \emph{caratteristiche} - ovvero proprietà osservabili e misurabili di un oggetto - in base alle quali decretare l'appartenenza all'una o all'altra classe, non è un problema banale.
            Si può intuire quanto sia vasto l'insieme delle caratteristiche valutabili nella rappresentazione di un oggetto.
            Ovviamente la natura della rappresentazione influisce nella scelta delle caratteristiche più rilevanti.
            
            Si ricordi che un'immagine di profondità rappresenta la realtà attraverso il valore della distanza misurata in ogni punto dello spazio osservato. 
            È naturale, quindi, considerare tali distanze come caratteristiche misurabili dell'oggetto rappresentato.

            In un sistema allenabile, sarà premura del modulo di apprendimento selezionare le caratteristiche più rilevanti per un oggetto.
            Senza anticipare altro sulla questione dell'apprendimento, è comunque utile fornire una descrizione in linguaggio naturale delle caratteristiche della forma del profilo umano, da tale visuale, che saltano immediatamente all'occhio.

            \begin{enumerate}
                \item L'immagine di una persona è caratterizzata da uno \emph{spazio vuoto}\footnote{Per \emph{spazio vuoto} si intende una regione di spazio il cui valore della distanza, percepita dal sensore, è molto vicino al quello della distanza del pavimento della stanza.} di fronte ad essa e dietro di essa.

                \item A sinistra della spalla sinistra ed a destra della spalla destra del profilo dall'alto di una persona, sono presenti degli spazi vuoti.
                
                \item Tra la testa e le spalle vi è una differenza di altezza.
            \end{enumerate}

            Si segua quindi la seguente convenzione: le immagini che soddisferanno le precedenti proprietà saranno chiamate immagini \emph{Head and Shoulders Profile}, o più brevemente \emph{immagini HASP}.

        \subsection{Flusso di Lavoro}
        \label{sub:overall_workflow}
            A conclusione di questa premessa, per evitare di perdersi in questo miscuglio di osservazioni eterogenee apparentemente prive di uno scopo preciso, è opportuno definire qual è il \emph{workflow} principale dell'intero sistema di rilevamento.
            Sostanzialmente quest'ultimo è diviso in due componenti accoppiati, uno delegato all'apprendimento delle caratteristiche di classificazione, l'altro delegato all'effettivo rilevamento.
            Come è facilmente intuibile, il secondo non può funzionare senza il risultato fornito dal primo, che è il più complesso e corposo dei due.

            \subsubsection{Allenamento}
                Per allenare il sistema è necessario creare un insieme di allenamento, ovvero un insieme i cui elementi sono delle immagini di profondità che ritraggono persone (immagini HASP) e non.
                In fase di creazione, ogni elemento di tale insieme viene dotato di un'etichetta che identifica la reale classe di appartenenza dell'oggetto.

                La componente software che si occupa dell'allenamento del sistema implementa l'algoritmo Adaboost.
                Quest'ultimo riceve in input l'insieme di allenamento, i cui elementi, dotati della rispettiva classificazione reale, sono alla base della scelta delle caratteristiche migliori per la descrizione delle classi di oggetti.
                Alla fine della sua esecuzione, Adaboost restituisce come output un classificatore.

                Nei capitoli successvi si darà una definizione più formale di quello che è un classificatore, per il momento è sufficiente una definizione intuitiva: un classificatore \emph{classifica} i vari oggetti, ovvero fornisce una \emph{previsione} della classe di appartenenza relativamente ad ognuno di essi.

                La classificazione effettuata da questa componente approssima solamente la classificazione reale. 
                La bontà di tale approssimazione sarà il parametro di valutazione della prestazione generale del sistema.

                Nel caso di Adaboost il classificatore risultante sarà simile ad una collezione di test: il risultato di tali test, eseguiti su di un qualsiasi oggetto, fornirà la previsione della classificazione dell'oggetto stesso.

            \subsubsection{Rilevamento}
                In questa fase il sistema analizza i frame di profondità delle acquisizioni in ordine sequenziale, alla ricerca di persone al suo interno.

                Il classificatore ottenuto al termine dell'esecuzione di Adaboost, sarà in grado di classificare porzioni di immagini di profondità, ma non è in grado di predire direttamente, a partire da un intero frame, la presenza e la posizione di una o più persone al suo interno: le porzioni analizzabili dal classificatore hanno dei vincoli dimensionali da rispettare, che in prima approssimazione si possono immaginare come dei quadrati di dimensione costante.
                L'attività di rilevamento della persona all'interno del frame, quindi, consterà della sequenziale analisi di tutte le porzioni di frame che rispettano tali vincoli, al fine di coprire l'intera area.

                Si vedrà in seguito che nei pressi di una persona nell'immagine di profondità, saranno molteplici le porzioni di frame per le quali il rilevamento darà esito positivo.
                Si pone quindi l'ulteriore problema di selezionare, delle tante porzioni che hanno dato esito positivo, quella che meglio approssima la reale posizione della persona. 


	% !TEX root=../index.tex



\chapter{Conclusioni}

% Questa parte non è stata ancora analizzata
\begin{appendices}
    \chapter{Software Sviluppato}
    \label{chap:software}
        \section{Componenti}
            \subsection{Creatore dei Dataset}
            \subsection{Allenamento}
            \subsection{Tuning, Testing, Rilevamento}
        \section{Tecnologie utilizzate}
            \subsection{C++ e Matlab}
            \subsection{Git e Github [Opzionale]}
        \section{Proposte di miglioramento}

    \chapter{Cenni del funzionamento del sensore Kinect}
    \label{chap:kinect_sensor}
\end{appendices}


	% Introduzione
	% % !TEX root=../index.tex

\chapter{Introduzione}
\label{cap:Introduzione}
\emph{Presentazione del problema, Stato dell'arte e descrizione delle caratteristiche peculiari del problema di riconoscimento.
Cenni sul funzionamento del sensore Kinect.
Head and Shoulder Profile.
Analisi del problema di classificazione.}

Il monitoraggio delle persone è un problema di visione artificiale di importanza fondamentale.
Sistemi di \emph{human sensing} vengono continuamente sviluppati ed utilizzati nei più disparati contesti applicativi. Sistemi di sorveglianza, apparecchiatura di supporto per missioni di salvataggio, dispositivi designati all'utilizzo in ambienti assistivi automatizzati e persino alcuni sistemi automatici per effettuare indagini di mercato utilizzano tecniche di percezione automatica delle persone dall'elaborazione dei dati acquisiti per mezzo di sensori.

Nel 2013 Zhu e Wong descrivono in \cite{Zhu13} un sistema allenabile di rilevamento e conteggio delle persone che attraversano una stanza.
Il riconoscimento della persona avviene elaborando i dati catturati dal sensore di profondità di un dispositivo Kinect, il quale è montato sul soffitto della stanza ed è puntato verso il pavimento. La crescente popolarità dei dispositivi Kinect, anche al di fuori degli ambienti videoludici, lo rende un interessante oggetto di studio.

\section{Head and Shoulder Profile}
\label{sec:hasp}
\begin{wrapfigure}{L}{0cm}
    \centering
    \includegraphics[width=5cm]{img/no_occlusion.png}
    \caption{Due persone in un ritaglio proveniente da un frame di profondità. Il frame è stato acquisito con il Kinect V1, a differenza di quello in figura \ref{fig:spatial_feature}, acquisito con il Kinect V2.}
    \label{fig:no_occlusion}
\end{wrapfigure}

In condizioni ottimali, la figura umana ripresa dall'alto è composta solamente dalla testa e dalle spalle.
Finchè si trova quasi in corrispondenza del sensore, nella parte centrale dell'immagine di profondità, la restante parte del corpo rimane quasi del tutto nascosta. Sulla posizione delle braccia non si possono fare delle assunzioni precise.

In figura \ref{fig:no_occlusion}, la figura dell'individuo a destra corrisponde a tale descrizione, mentre dell'altro è visibile parte del corpo ed una delle due spalle è nascosta dalla testa.
Nelle zone periferiche di un frame di profondità, l'immagine è soggetta alla \emph{distorsione prospettica} così come lo è quella di qualunque telecamere RGB.

Tale distorsione costituisce un disturbo, dal momento che la figura dello stesso soggetto varia a seconda della relativa posizione nell'area di visualizzazione.
Si vedrà più avanti come affrontare tale situazione, per il momento si considerano solamente le immagini proveniente dalle zone centrali dei frame (come quella in figura \ref{fig:spatial_feature}).

Un grande vantaggio dell'utilizzo del Kinect in questa configurazione è l'\emph{assenza di occlusione}.
Infatti, rispetto ai molteplici sistemi di riconoscimento frontali, i soggetti non possono nascondersi l'uno con l'altro al sensore.

Utilizzare le immagini di profondità significa ragionare con le distanze: piuttosto che cercare di descrivere l'immagine del profilo umano in termini di forma, deve essere descritto in termini di differenze di quota rispetto all'ambiente circostante.
Alla luce di ciò, si possono identificare i seguenti criteri descrittivi:

\begin{enumerate}
    \item L'immagine di una persona è caratterizzata da uno spazio vuoto di fronte ad essa e dietro di essa. Per \emph{spazio vuoto} si intende una regione la cui distanza dal sensore è circa quella del pavimento.
    \item A sinistra della spalla sinistra ed a destra della spalla destra del profilo dall'alto di una persona, sono presenti degli spazi vuoti.
    \item Tra la testa e le spalle vi è una differenza di altezza.

\end{enumerate}

\begin{wrapfigure}{R}{0cm}
    \centering
    \includegraphics[width=4cm]{img/spatial_features.png}
    \caption{Una persona mentre cammina}
    \label{fig:spatial_feature}
\end{wrapfigure}

Di qui in avanti, con l'acronimo \emph{HASP} (\emph{Head And Shoulder Profile}), ci si riferirà proprio al profilo della persona ripreso dall'alto che soddisfa i criteri appena elencati.


	% Questa sezione va accorpata a quella precedente
	% % !TEX root=../index.tex

\chapter{Panoramica del sistema}
\label{cap:overview}
Riproducendo la configurazione proposta in \cite{Zhu13} al soffitto del laboratorio è stato fissato un dispositivo Kinect V2 rivolto verso il pavimento: all'ingresso di una persona nella visuale del Kinect, ne saranno ben visibili la testa e le spalle.

\section{Il Kinect}
\label{sec:sensor}
Ad onor del vero, ciò che viene comunemente chiamato \emph{sensore di profondità} (o sensore di distanza) del Kinect, è in realtà uno \emph{scanner 3D a luce strutturata}\footnote{Nonostante ciò, lo si continuerà a chiamare \emph{sensore di profondità}, leggermente inesatto, ma decisamente più breve ed intuitivo}.
\begin{wrapfigure}{R}{0cm}
    \centering
    \includegraphics[width=8cm]{img/3d-structured-light-scanner.png}
    \label{fig:structured_light_scanner}
    \caption{Schematizzazione di uno scanner 3D a luce strutturata.}
\end{wrapfigure}
Un sorgente di raggi infrarossi proietta una serie di pattern codificati. La deformazione indotta dalle superfici degli oggetti interessati viene acquisita da una o più telecamere ed utilizzata per il calcolo delle coordinate tridimensionali.

Il risultato di un sensore di questo tipo è un insieme di triplette $(x,y,z)$, organizzate in una \emph{immagine di profontità}, una struttura dati che è molto simile ad una semplice immagine in scala dei grigi, dove il valore di ogni pixel rappresenta la misura in millimetri della distanza della superficie dal sensore.
La forte somiglianza con le immagini in scala dei grigi è supportata dal fatto che ogni pixel è codificato utilizzando 16bit.

La massima affidabilità del sensore del Kinect si ha per distanza comprese tra $50cm$ e $4,5m$.
Il dispositivo è montato al soffitto a $2,8m$ da terra e ha un campo visivo di $70^{\circ} \times 60^{\circ}$, il quale, all'altezza del pavimento, determina un'area di cattura di circa $4m \times 5m$.

La dimensione di ogni immagine di profondità è di $512 \times 424$ pixel. Nativamente non vengono codificate in alcun modo particolare, sono delle semplici matrici di interi.
Il Kinect V2 è in grado di catturarne fino a 30 al secondo. Utilizzando un apposito software di registrazione è stato possibile mettere insieme dei video di profondità a 30 fps.


\section{Head and Shoulder Profile}
\label{sec:hasp}

\begin{wrapfigure}{L}{0cm}
    \centering
    \includegraphics[width=5cm]{img/no_occlusion.png}
    \caption{Due persone in un ritaglio proveniente da un frame di profondità. Il frame è stato acquisito con il Kinect V1, a differenza di quello in figura \ref{fig:spatial_feature}, acquisito con il Kinect V2.}
    \label{fig:no_occlusion}
\end{wrapfigure}

In condizioni ottimali, la figura umana ripresa dall'alto è composta solamente dalla testa e dalle spalle.
Finchè si trova quasi in corrispondenza del sensore, nella parte centrale dell'immagine di profondità, la restante parte del corpo rimane quasi del tutto nascosta. Sulla posizione delle braccia non si possono fare delle assunzioni precise.

In figura \ref{fig:no_occlusion}, la figura dell'individuo a destra corrisponde a tale descrizione, mentre dell'altro è visibile parte del corpo ed una delle due spalle è nascosta dalla testa.
Nelle zone periferiche di un frame di profondità, l'immagine è soggetta alla \emph{distorsione prospettica} così come lo è quella di qualunque telecamere RGB.

Tale distorsione costituisce un disturbo, dal momento che la figura dello stesso soggetto varia a seconda della relativa posizione nell'area di visualizzazione.
Si vedrà più avanti come affrontare tale situazione, per il momento si considerano solamente le immagini proveniente dalle zone centrali dei frame (come quella in figura \ref{fig:spatial_feature}).

Un grande vantaggio dell'utilizzo del Kinect in questa configurazione è l'\emph{assenza di occlusione}.
Infatti, rispetto ai molteplici sistemi di riconoscimento frontali, i soggetti non possono nascondersi l'uno con l'altro al sensore.

Utilizzare le immagini di profondità significa ragionare con le distanze: piuttosto che cercare di descrivere l'immagine del profilo umano in termini di forma, deve essere descritto in termini di differenze di quota rispetto all'ambiente circostante.
Alla luce di ciò, si possono identificare i seguenti criteri descrittivi:

\begin{enumerate}
    \item L'immagine di una persona è caratterizzata da uno spazio vuoto di fronte ad essa e dietro di essa. Per \emph{spazio vuoto} si intende una regione la cui distanza dal sensore è circa quella del pavimento.
    \item A sinistra della spalla sinistra ed a destra della spalla destra del profilo dall'alto di una persona, sono presenti degli spazi vuoti.
    \item Tra la testa e le spalle vi è una differenza di altezza.

\end{enumerate}

\begin{wrapfigure}{R}{0cm}
    \centering
    \includegraphics[width=4cm]{img/spatial_features.png}
    \caption{Una persona mentre cammina}
    \label{fig:spatial_feature}
\end{wrapfigure}

Di qui in avanti, con l'acronimo \emph{HASP} (\emph{Head And Shoulder Profile}), ci si riferirà proprio al profilo della persona ripreso dall'alto che soddisfa i criteri appena elencati.


\section{Un Problema di Classificazione}
\label{sec:classification_problem}
Quello che il sistema di rilevamento deve fare è riconoscere se e dove è presente l'immagine HASP di una persona all'interno del frame di profondità.
Ciò avviene analizzando sequenzialmente delle porzioni del frame originale in modo da coprire tutta l'area. Si tornerà più avanti su questo aspetto, per il momento si consideri che il problema di rilevamento è naturalmente riconducibile ad un problema di classificazione.

Gli oggetti da classificare sono immagini di profondità e le possibili classi di appartenenza di tali oggetti sono due: la classe delle immagini che ritraggono il profilo di una persona è la classe delle immagini che invece ne sono prive.
Il fatto che vi siano solamente due classi lo rende un \emph{problema di classificazione binario}.

L'operazione di classificazione avviene mediante la misurazione di alcune \emph{caratteristiche} (\emph{feature}) interessanti dell'immagine di profondità.
I criteri di riconoscimento presentati nella sezione \ref{sec:hasp} sono dei validi esempi di caratteristiche. Sono espresse in linguaggio naturale e sono comprensibili agli esseri umani, ma mancano di rigore e perciò non sono direttamente implementabili come criteri di classificazione delle immagini HASP.

Il \emph{classificatore} è il componente che esegue effettivamente la classificazione. La definizione di un classificatore può avvenire \emph{manualmente}, mediante l'applicazione di un modello atto a descrivere al meglio gli oggetti di una determinata classe, oppure può essere \emph{allenato}.
Gli algoritmi per la definizione di classificatori allenati si distinguono a loro volta tra algoritmi di \emph{allenamento supervisionato} e di \emph{allenamento non supervisionato}, a seconda che facciano uso o meno di un insieme di allenamento.

Un insieme di allenamento non è altro che un insieme di oggetti adeguati al problema di classificazione per cui viene fornita la classificazione \emph{reale}.
In questo caso saranno delle immagini di profondità opportunamente marcate come immagini che contengono o meno il profilo di una persona.

Il sistema in esame prevede l'utilizzo di classificatori allenati con \emph{Adaboost}. Sono molti i sistemi di riconoscimento di immagini che utilizzano Adaboost per allenare i rispettivi classificatori, primo tra tutti il quello di riconoscimento facciale proposto da Viola e Jones \cite{Viola04}, ad oggi considerato il più robusto ed efficiente della sua categoria.

La forte somiglianza con il lavoro di Viola e Jones farà sì che saranno molteplici i riferimenti ad esso nel testo e i dettagli implementativi in comune.


\section{Preprocessing}
\label{sec:preprocessing}
Ogni pixel di un'immagine di profondità denota la distanza in millimetri della superficie dell'oggetto dal sensore. Tuttavia, prima di poter essere utilizzate da qualsiasi componente del sistema di rilevamento, occorre convertire il valore di ogni pixel in quota della superficie dal pavimento.

Il pavimento è la superficie la cui distanza dal sensore è massima, quindi per individuarne il valore sarebbe sufficiente ricercare il valore massimo di ciascuna immagine di profondità.
Nella pratica ciò non è vero, in quanto, a causa della distorsione prospettica, nelle aree periferiche dell'immagine la distanza percepita del pavimento è maggiore rispetto a quella misurata al centro dell'immagine.

Quindi, per determinare la reale distanza del pavimento è necessario disporre di un'immagine di riferimento che ritragga solamente il pavimento stesso. A questo punto la distanza di riferimento è quella misurata esattamente al centro del frame.

La trasformazione delle distanze in quote avviene, come è intuibile, sostituendo il valore di ogni pixel $d$ con il valore $d'$.
\begin{equation}
    d' = d - d_{pavimento}
    \label{eq:floor_distance}
\end{equation}
L'operazione di conversione ha complessità computazionale $\Theta(n \cdot m)$, dove $n$ ed $m$ sono le dimensioni del frame.


	% % !TEX root=../index.tex

\section{Weak learner}
\label{sec:weak_learner}
\begin{enumerate}
    \item Problema di classificazione
    \item Caratteristiche
\end{enumerate}

\subsection{Feature Haar-like}
\label{sub:feature_haar_like}

\subsubsection{Immagine Integrale}
\label{subs:immagine_integrale}

\subsection{Decision Stump}
\label{subs:decision_stump}


	% \input{section/adaboost.tex}

	% \chapter{Tuning}
	% \label{chap:Tuning}
	% \emph{Definizione dei criteri di valutazione delle prestazioni.
	% Algoritmo di massimizzazione dell'accuratezza.
	% Valutazione del risultato dell'algoritmo di apprendimento.
	% Valutazione degli elementi di disturbo.}
	%
	% \chapter{Rilevamento}
	% \label{chap:Rilevamento}
	% \emph{Presentazione della tecnica di rilevamento su registrazioni reali.
	% Algoritmo evoluto di selezione della finestra ottima.
	% Tecniche di ottimizzazione dell'operazione di rilevamento a regime.
	% Presentazione e confronto dei risultati ottenuti con quelli in letteratura.}
	%
	% \chapter{Conclusioni}
	% \label{chap:Conclusioni}
	%
	% \begin{appendices}
	%
	% 	% !TEX root=../index.tex

\chapter{Software Sviluppato}
\label{chap:software}
    \section{Componenti}
        \subsection{Gestore dei Dataset} % (fold)
        \label{sub:training_set_creator}
            \subsubsection{Interfaccia Grafica}
            \subsubsection{Strutture Dati per la persistenza}
            \subsubsection{Tecnologie utilizzate}        
        % subsection training_set_creator (end)
        
        \subsection{Allenamento}
            \subsubsection{Punti critici dal punto di vista computazionale}
            \subsubsection{Mex files}
            \subsubsection{Architettura delle librerie}
            \subsubsection{Struttura dati per la persistenza}
        \subsection{Tuning, Testing, Rilevamento}
            \subsubsection{Organizzazione degli script e delle funzioni}
            \subsubsection{Struttura dati per la persistenza}
    \section{Tecnologie utilizzate}
        \subsection{C++ e Matlab}
            \subsubsection{Matlab: prototipazione e componenti non critiche}
            \subsubsection{C++: ottimizzazione delle componenti critiche}
        \subsection{Git e Github [Opzionale]}
            \subsubsection{Sistemi VCS}
    \section{Proposte di miglioramento}
        \subsubsection{Componenti da ottimizzare}
        \subsubsection{Nuovi linguaggi da utilizzare}
	%
	% 	% !TEX root=../index.tex

\chapter{Accenni sul Funzionamento e le Caratteristiche del Sensore del Kinect} % (fold)
\label{chap:kinect}

Ciò che viene comunemente chiamato \emph{sensore di profondità} o \emph{sensore di distanza}, in riferimento al dispositivo Kinect, è in realtà uno \emph{scanner 3D a luce strutturata}.

Un sorgente di raggi infrarossi proietta una serie di pattern codificati nello spazio.
Le superfici colpite inducono una deformazione nella struttura di tali pattern.
Il pattern deformato viene quindi catturato da una o più telecamere, che confrontando la deformazione con il pattern originario, riescono a ricostruire, dalla rappresentazione bidimensionale di ogni punto nello spazio, le sue coordinate tridimensionali.

Il risultato di un sensore di questo tipo è un insieme di triplette $(x,y,z)$, organizzate in una \emph{immagine di profondità}, una struttura dati che è molto simile ad una semplice immagine in scala dei grigi\footnote{La forte somiglianza con le immagini in scala dei grigi è supportata dal fatto che ogni pixel è codificato utilizzando 16bit.}, dove il valore di ogni pixel rappresenta la misura in millimetri della distanza della superficie dal sensore.

La massima affidabilità del sensore del Kinect V2 si ha per distanza comprese tra $50cm$ e $4,5m$.
Il dispositivo è montato al soffitto a $2,8m$ da terra e ha un campo visivo di $70^{\circ} \times 60^{\circ}$, il quale, all'altezza del pavimento, determina un'area di cattura di circa $4m \times 5m$.

La dimensione di ogni immagine di profondità è di $512 \times 424$ pixel. 
Nativamente non vengono codificate in alcun modo particolare, sono delle semplici matrici di interi.
È in grado di catturarne fino a 30 al secondo. 
Utilizzando un apposito software di registrazione è stato possibile mettere insieme dei video di profondità a 30 fps.

% chapter kinect (end)
	%
	% \end{appendices}

	% Far comparire tutti gli elementi della bibliografia
	\nocite{*}
	\bibliographystyle{plainnat}
	\bibliography{bibliografia}{}

\end{document}
