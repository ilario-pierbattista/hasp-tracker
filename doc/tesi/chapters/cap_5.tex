%!TEX root=../index.tex

\chapter{Rilevamento}
\label{chap:rilevamento}
    \section{Tecnica di Rilevamento}
    \label{sec:detection_tecnique}
        \subsection{Detection Window} % (fold)
        \label{sub:detection_window}
            Vengono create delle \emph{detection window} per gestire il rilevamento su frames.
            La detection window viene fatta scorrere sequenzialmente su tutta l'area del frame, in modo da coprirne tutta l'area.
        % subsection detection_window (end)
        
        \subsection{Resize della Detection Window} % (fold)
        \label{sub:resize_della_detection_window}
            A causa della variabilità delle dimensioni della figura HASP della persona, si rende necessario poter effettuare dei ridimensionamenti della finestra di rilevamento. 
            Per ridimensionarla bisogna ridimensionare tutte le feature di haar associate ai classificatori della finestra.
        % subsection resize_della_detection_window (end)

        \subsection{Sliding della Detection Window} % (fold)
        \label{sub:sliding_della_detection_window}
            Si fa scorrere la finestra in modo da coprire tutta l'area del frame.
            Ciò permette di effettuare delle rilevazioni in contesti statici (su singoli frame o su frame non correlati tra di loro).
            Quando si ha a che fare con frame sequenziali, è possibile evitare di analizzare tutte le parti del frame in condizioni di regime, rendendo il processo più veloce.

     
            \subsubsection{Transitorio} % (fold)
            \label{ssub:transitorio}
                All'inizio di una rilevazione, il sistema è \emph{cieco}: deve analizzare completamente tutta l'area del frame.
            
            % subsubsection transitorio (end)

            \subsubsection{A Regime} % (fold)
            \label{ssub:a_regime}
                Successivamente ad una prima scansione iniziale del frame, è possibile ridurre l'area da scandire per i rilevamenti successivi.
                Se è stato rilevato un umano, si procede ad analizzare solamente la porzione di aree nelle sue immediate vicinanze.
                In ogni caso, si continua ad analizzare il bordo del frame, in attesa di nuovi ingressi.
            
            % subsubsection a_regime (end)

        % subsection sliding_della_detection_window (end)

    \section{Selezione della Finestra Migliore} %@TODO
    \label{sec:best_detection_window}
        % \subsection{Introduzione al problema}
            In corrispondenza di una persona, ci saranno più finestre di rilevamento a dare esito positivo, per cui è necessario trovare un meccanismo per la selezione dell'area esatta all'interno della quale si trova la persona.
        
        % \subsection{Algoritmo di selezione}
            In prima approssimazione è stato sviluppato un algoritmo basato sul calcolo del baricentro delle finestre.
            Ogni finestra aveva lo stesso peso.

            In un secondo momento le finestre sono state pesate con valori decrescenti man mano che queste si allontanano dalla posizione della persona allo stato precedente.
            Finestre molto distanti, difficilmente portano a risultati veritieri, quindi vengono premiate le finestre di rilevamento nelle immediate vicinanze della persona.

    \section{Confronto con l'Algoritmo G-C} %@TODO
    \label{sec:gc_algorithm_comparison}