%!TEX root=../index.tex

\chapter{Conclusioni} % (fold)
\label{cha:conclusioni}
	A partire da quanto descritto da Zhu e Wong in \cite{Zhu13} è stata elaborata una semplice implementazione del loro sistema di human sensing.

	Nei capitoli precedenti sono state largamente commentate le molteplici somiglianze con il sistema di \emph{face recognition} di Viola e Jones: l'uso dello stesso algoritmo di allenamento, della stessa famiglia di feature e quello di alcuni parametri caratteristici, come la dimensione delle immagini di allenamento.

	L'impiego di immagini di profondità, che forniscono una rappresentazione essenziale della realtà, fondamentalmente priva della ricchezza di dettagli riscontrabile in immagini RGB, comporta l'elaborazione di classificatori molto più semplici rispetto a quelli per il riconoscimento dei volti e quindi, a differenza di questi ultimi, non necessitano di particolari accorgimenti per velocizzare l'esecuzione.

	La valutazione dell'accuratezza del sistema ed il confronto dei risultati ottenuti con quelli esposti in letteratura, confermano la correttezza dell'implementazione e la complessiva qualità del sistema.

	Rimangono migliorabili le prestazioni in fase di rilevamento.
	La logica utilizzata per la selezione delle finestre di rilevamento ottimali ha validità limitata al caso in cui è presente solo una persona nell'area del frame, il che costituisce un limite importante, ma raggirabile implementando una logica di selezione migliore.


% chapter conclusioni (end)