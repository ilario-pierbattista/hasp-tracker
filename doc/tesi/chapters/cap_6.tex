%!TEX root=../index.tex

\chapter{Conclusioni} % (fold)
\label{cha:conclusioni}
	Si è partiti dalla descrizione fornita da Zhu e Wong del sistema di human sensing.
	Da essa è partito un lungo lavoro di documentazione, senza il quale non sarebbe stato possibile avere pienamente coscienza di ogni particolare, di ogni sfumatura di quest'ultimo.

	L'implementazione proposta è semplice, ma è stata progettata per poter accogliere, con poco sforzo, eventuali miglioramenti futuri.
	Sono stati utilizzati strumenti moderni, sia nello sviluppo che nella sperimentazione, e dove mancavano ne sono stati creati di nuovi.

	Nei test, livelli di accuratezza molto buoni ed il comportamento complessivo su dati reali è discreto.
	Vi sono ancora numerosi punti su cui è possibile agire per migliorare le prestazioni, sia in velocità d'esecuzione che in accuratezza.
	Lo sviluppo del sistema non è stato tanto lineare quanto il suo resoconto: ad un certo punto è dovuto terminare.

	Quel che si ha, in definitiva, è l'implementazione di un framework di \emph{object detection} maturo e stabile, specializzata nel riconoscimento di persone attraverso immagini di profondità.

% chapter conclusioni (end)