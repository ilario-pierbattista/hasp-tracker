% !TEX root=../index.tex

\chapter{Introduzione}
\label{chap:introduction}
    \section{Human Sensing}
    \label{sec:human_sensing}
        L'insieme di tecniche e di soluzioni per il riconoscimento della presenza della persona nello spazio prende il nome \emph{human sensing}.

        Vengono utilizzati vari dispositivi di sensoristica per l'acquisizione di informazioni ai fini del riconoscimento. 
        Dall'elaborazione delle informazioni acquisite, i sistemi di human sensing sono in grado di rilevare la presenza e la posizione della persona all'interno dell'ambiente interessato.

        I contesti applicativi di sistemi di questo genere sono variegati ed in costante espazione.
        Sistemi per il rilevamento delle persone si adattano perfettamente in contesti di sorveglianza, sia al fine di garantire la sicurezza di un ambiente in termini di rilevamento delle intrusioni, sia al fine di costruire delle soluzioni di monitoraggio in ambienti assistivi automatizzati.
        Dispositivi in grado di localizzare corpi umani sono utilissimi nelle attività \emph{search \& rescue}, dove è necessario, in caso di catastrofi o calamità naturali, localizzare nel minor tempo possibile i superstiti in condizioni avverse.
        Vi sono delle applicazioni anche in ambito economico: soluzioni di \emph{people counting} sono sempre più usate nei processi di \emph{business intelligence} per effettuare analisi di mercato.

        Molto spesso, la natura dei sensori utilizzati per le acquisizioni, fanno in modo che i problemi di human sensing si intersechino con problemi di \emph{computer vision}: l'utilizzo di sensori ad acquisizione visiva comporta l'insorgere di svariati problemi di visione artificiale.

        Lo scopo della visione artificiale è quello di riprodurre la vista umana, intesa non come semplice acquisizione di rappresentazione bidimensionale di una regione di spazio, ma mira ad una reale interpretazione del relativo contenuto.

        \subsection{Stato dell'Arte}
            Il crescente interesse per le soluzioni di human sensing ha dato luogo a numerosi studi e ricerche sull'argomento.
            Rimanendo nel contesto del rilevamento legato a problemi di computer vision, spiccano i seguenti risultati.

            \begin{description}
                \item[People Tracking and Counting] \citet{Papageorgiou98} nel \citeyear{Papageorgiou98} proposero una tecnica di riconoscimento e conteggio di pedoni a partire da immagini RGB.
                Elaborando opportunamente le informazioni raccolte da comuni telecamere, il sistema era in grado di apprendere a riconoscere il pattern della figura umana analizzando le variazioni di intensità tra le diverse aree delle immagini.
                Per l'apprendimento utilizzarono una \emph{macchina a vettori di supporto}\footnote{Consultare la sezione \ref{sec:supervised_ensamble_learning} ed in generale tutto il capitolo \ref{chap:adaboost} per ulteriori informazioni al riguardo.}.
                
                \item[Face Recognition] Viola e Jones in \cite{Viola04} descrivono un framework di \emph{object recognition}, applicandolo con successo al caso del riconoscimento dei volti.
                Ancora una volta, si tratta di un sistema allenabile che utilizza l'algoritmo \emph{Adaboost}\footnote{Per maggiori informazioni consultare \cite{Freund97} ed il capitolo \ref{chap:adaboost}.}.

                \item[Human Sensing tramite Kinect] Anche il dispositivo Kinect, prodotto da Microsoft, costituisce, nel complesso, un valido esempio di sistema di human sensing.
                La quantità e la qualità dei sensori con cui è equipaggiato, il costo relativamente basso e la disponibilità di \emph{SDK} e toolkit integrabili con ambienti di sviluppo evoluti, stanno accrescendo la popolarità di tale dispositivo al di là delle applicazioni videoludiche.
                Cresce anche l'interesse per i \emph{serious game}, videogiochi sviluppati con una doppia valenza, quella di assistere e guidare i pazienti in contesti riabilitativi.
            \end{description}

    \section{Panoramica Generale}
    \label{sec:overview}
        Questo elaborato descrive principalmente il lavoro svolto per l'implementazione del sistema di rilevamento descritto da Zhu e Wong in \cite{Zhu13}.
        È correlato di numerosi approfondimenti teorici, necessari, sia per avere una corretta visione d'insieme del problema, sia per essere in grado di agire efficacemente nella futura richiesta di miglioramenti nell'implementazione proposta.

        \subsection{Il Lavoro di Zhu e Wong} % (fold)
        \label{sub:il_lavoro_di_zhu_e_wong}
            Nel \citeyear{Zhu13} Zhu e Wong presentano un sistema di riconoscimento della figura umana, ripresa dall'alto di una stanza, con il dispositivo Kinect.
            Si tratta di una soluzione di human sensing, intercalata in una particolare condizione ambientale, che utilizza il \emph{sensore di profondità} del dispositivo Kinect come sorgente per l'acquisizione di dati.

            Come altre soluzioni proposte in precedenza, il sistema deve essere allenato a riconoscere persone.
            L'algoritmo utilizzato per lo sviluppo dei criteri di riconoscimento è \emph{Adaboost}, lo stesso utilizzato per il framework di \emph{face detection} sviluppato da Viola e Jones.
            Come si potrà vedere in seguito, in maniera più dettagliata, saranno numerose le somiglianze con quest'ultimo.

        \subsection{Configurazione Hardware}
        \label{sub:hardware_configuration}
            Precisamente la versione del dispositivo Kinect utilizzato è la \emph{V2}.
            Il sensore di profondità del Kinect V2 fornisce una rappresentazione bidimensionale dello spazio, fondamentalmente sotto forma di immagini particolari. 
            In queste ultime ogni pixel corrisponde il valore in millimetri della distanza dal sensore della superficie dell'oggetto presente nell'immagine.
            Ci si riferirà ad esse chiamandole \emph{immagini di profondità}.

            Il sensore del Kinect, di cui è disponibile una piccola descrizione più dettagliata all'appendice \ref{chap:kinect}, fornisce uno stream di tali immagini ad una frequenza di 30 frame al secondo: è possibile quindi registrare dei \emph{video di profondità}.

            \subsubsection{Visuale Top-Down}
                Il dispositivo Kinect viene montato al soffitto di una stanza e l'ambiente viene ripreso da tale prospettiva.
                Solitamente l'altezza a cui viene montato è di poco inferiore alla distanza del soffitto dal pavimento (poco meno di $3m$), inoltre la linea focale del sensore dovrebbe essere quanto più possibile ortogonale al pavimento della stanza, in modo da ridurre ai minimi termini la presenza di asimmetrie nelle riprese.
                
                Tali distanze sono perfettamente compatibili con le specifiche tecniche del dispositivo stesso.
                Nel caso in cui vi sia la necessità di montare il Kinect a soffitti più alti di $4m$, si possono utilizzare delle lenti correttive per aumentare il range di affidabilità del sensore.

                Molto sistemi di riconoscimento utilizzano il Kinect - e più in generale, qualsiasi dispositivo di acquisizione visivo - in posizione frontale rispetto ai soggetti da riconoscere.
                Di fatti, il dispositivo, concepito per applicazioni videoludiche, è progettato per operare in tale posizione.
                Tuttavia, la configurazione descritta precedentemente, ha l'enorme vantaggio di eliminare l'occlusione del soggetto: in tali condizioni una persona non può nasconderne dietro di sè un'altra alla vista del sensore (se non in scomode posizioni), cosa invece frequentissima nei i sistemi di rilevamento frontali.

        \subsection{\emph{Head and Shoulders Profile}}
        \label{sub:hasp}
            Riconoscere un oggetto significa mettere in relazione la sua rappresentazione con un concetto, più o meno specifico, che lo descriva.
            In questo caso, dovendo riconoscere delle persone, si dovrà mettere correttamente in relazione un'immagine di profondità che raffigura un individuo con il \emph{concetto di persona}, mentre per un'immagine che non la raffigura, si dovrà evitare una relazione di tale natura.
            
            Si definiscono due classi.
            Il concetto di classe è molto simile a quello delle \emph{classi di equivalenza}, relativo all'algebra astratta, ma anche a quello delle classi come prototipi di \emph{oggetti software}, relativo al paradigma di programmazione ad oggetti.
            Una classe costituisce un insieme di oggetti che condividono determinate proprietà caratteristiche: 
            per le classi di equivalenza, tutti gli elementi appartenenti ad essa devono essere tra di loro in una relazione di equivalenza;
            per le classi software, tutte istanze di tali classi sono caratterizzate dalla medesima struttura (stessi attributi, stessi metodi), anche se poi ogni oggetto è ben distinto dall'altro. 

            Distinguere gli oggetti che rappresentano persone da quelli che non le rappresentano, significa classificare tali oggetti in due classi: la \emph{classe delle persone} e quella delle \emph{non persone}.
            A questo punto, bisogna iniziare a chiedersi quali debbano essere le proprietà degli elementi dell'una e quali quelli dell'altra.

            Bisogna sottolineare il fatto che gli oggetti di una stessa classe, hanno sì delle proprietà in comune tra di loro, ma averne altre per cui differiscono.
            Individuare le \emph{caratteristiche} - ovvero proprietà osservabili e misurabili di un oggetto - in base alle quali decretare l'appartenenza all'una o all'altra classe, non è un problema banale.
            Si può intuire quanto sia vasto l'insieme delle caratteristiche valutabili nella rappresentazione di un oggetto.
            Ovviamente la natura della rappresentazione influisce nella scelta delle caratteristiche più rilevanti.
            
            Si ricordi che un'immagine di profondità rappresenta la realtà attraverso il valore della distanza misurata in ogni punto dello spazio osservato. 
            È naturale, quindi, considerare tali distanze come caratteristiche misurabili dell'oggetto rappresentato.

            In un sistema allenabile, sarà premura del modulo di apprendimento selezionare le caratteristiche più rilevanti per un oggetto.
            Senza anticipare altro sulla questione dell'apprendimento, è comunque utile fornire una descrizione in linguaggio naturale delle caratteristiche della forma del profilo umano, da tale visuale, che saltano immediatamente all'occhio.

            \begin{enumerate}
                \item L'immagine di una persona è caratterizzata da uno \emph{spazio vuoto}\footnote{Per \emph{spazio vuoto} si intende una regione di spazio il cui valore della distanza, percepita dal sensore, è molto vicino al quello della distanza del pavimento della stanza.} di fronte ad essa e dietro di essa.

                \item A sinistra della spalla sinistra ed a destra della spalla destra del profilo dall'alto di una persona, sono presenti degli spazi vuoti.
                
                \item Tra la testa e le spalle vi è una differenza di altezza.
            \end{enumerate}

            Si segua quindi la seguente convenzione: le immagini che soddisferanno le precedenti proprietà saranno chiamate immagini \emph{Head and Shoulders Profile}, o più brevemente \emph{immagini HASP}.

        \subsection{Flusso di Lavoro}
        \label{sub:overall_workflow}
            A conclusione di questa rilassata premessa, per evitare di perdersi in questo miscuglio di osservazioni eterogenee apparentemente prive di uno scopo preciso, è opportuno definire qual è il \emph{workflow} principale dell'intero sistema di rilevamento.
            Sostanzialmente quest'ultimo è diviso in due componenti accoppiati, uno delegato all'apprendimento delle caratteristiche di classificazione, l'altro delegato all'effettivo rilevamento.
            Come è facilmente intuibile, il secondo non può funzionare senza il risultato fornito dal primo, che è il più complesso e corposo dei due.

            \subsubsection{Allenamento}
                Per allenare il sistema è necessario creare un insieme di allenamento, ovvero un insieme i cui elementi sono delle immagini di profondità che ritraggono persone (immagini HASP) e non.
                In fase di creazione, ogni elemento di tale insieme viene dotato di un'etichetta che identifica la reale classe di appartenenza dell'oggetto.

                La componente software che si occupa dell'allenamento del sistema implementa l'algoritmo Adaboost.
                Quest'ultimo riceve in input l'insieme di allenamento, i cui elementi, dotati della rispettiva classificazione reale, sono alla base della scelta delle caratteristiche migliori per la descrizione delle classi di oggetti.
                Alla fine della sua esecuzione, Adaboost restituisce come output un classificatore.

                Nei capitoli successvi si darà una definizione più formale di quello che è un classificatore, per il momento è sufficiente una definizione intuitiva: un classificatore \emph{classifica} i vari oggetti, ovvero fornisce una \emph{previsione} della classe di appartenenza relativamente ad ognuno di essi.

                La classificazione effettuata da questa componente approssima solamente la classificazione reale. 
                La bontà di tale approssimazione sarà il parametro di valutazione della bontà generale del sistema.

                Nel caso di Adaboost il classificatore risultante sarà simile ad una collezione di test: il risultato di tali test, eseguiti su di un qualsiasi oggetto, fornirà la previsione della classificazione dell'oggetto stesso.

            \subsubsection{Rilevamento}
                In questa fase il sistema analizza i frame di profondità delle acquisizione in ordine sequenziale, alla ricerca di persone al suo interno.

                Il classificatore ottenuto al termine dell'esecuzione di Adaboost, sarà in grado di classificare porzioni di immagini di profondità, ma non è in grado di predire direttamente, a partire da un intero frame, la presenza e la posizione di una o più persone al suo interno: le porzioni analizzabili dal classificatore hanno dei vincoli dimensionali da rispettare, che in prima approssimazione si possono immaginare come dei quadrati di dimensione costante.
                L'attività di rilevamento della persona all'interno del frame, quindi, consterà della sequenziale analisi di tutte le porzioni di frame che rispettano tali vincoli, al fine di coprire l'intera area.

                Si vedrà in seguito che nei pressi di una persona nell'immagine di profondità, saranno molteplici le porzioni di frame per le quali il rilevamento darà esito positivo.
                Si pone quindi l'ulteriore problema di selezionare, delle tante porzioni che hanno dato esito positivo, quella che meglio approssima la reale posizione della persona. 
