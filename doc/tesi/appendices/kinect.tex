% !TEX root=../index.tex

\chapter{Accenni sul Funzionamento e le Caratteristiche del Sensore del Kinect} % (fold)
\label{chap:kinect}

Ciò che viene comunemente chiamato \emph{sensore di profondità} o \emph{sensore di distanza}, in riferimento al dispositivo Kinect, è in realtà uno \emph{scanner 3D a luce strutturata}.

Un sorgente di raggi infrarossi proietta una serie di pattern codificati nello spazio.
Le superfici colpite inducono una deformazione nella struttura di tali pattern.
Il pattern deformato viene quindi catturato da una o più telecamere, che confrontando la deformazione con il pattern originario, riescono a ricostruire, dalla rappresentazione bidimensionale di ogni punto nello spazio, le sue coordinate tridimensionali.

Il risultato di un sensore di questo tipo è un insieme di triplette $(x,y,z)$, organizzate in una \emph{immagine di profondità}, una struttura dati che è molto simile ad una semplice immagine in scala dei grigi\footnote{La forte somiglianza con le immagini in scala dei grigi è supportata dal fatto che ogni pixel è codificato utilizzando 16bit.}, dove il valore di ogni pixel rappresenta la misura in millimetri della distanza della superficie dal sensore.

La massima affidabilità del sensore del Kinect V2 si ha per distanza comprese tra $50cm$ e $4,5m$.
Il dispositivo è montato al soffitto a $2,8m$ da terra e ha un campo visivo di $70^{\circ} \times 60^{\circ}$, il quale, all'altezza del pavimento, determina un'area di cattura di circa $4m \times 5m$.

La dimensione di ogni immagine di profondità è di $512 \times 424$ pixel. 
Nativamente non vengono codificate in alcun modo particolare, sono delle semplici matrici di interi.
È in grado di catturarne fino a 30 al secondo. 
Utilizzando un apposito software di registrazione è stato possibile mettere insieme dei video di profondità a 30 fps.

% chapter kinect (end)